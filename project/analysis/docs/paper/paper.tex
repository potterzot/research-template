% AER-Article.tex for AEA last revised 22 June 2011
\documentclass[AER]{AEA}

% The mathtime package uses a Times font instead of Computer Modern.
% Uncomment the line below if you wish to use the mathtime package:
%\usepackage[cmbold]{mathtime}
% Note that miktex, by default, configures the mathtime package to use commercial fonts
% which you may not have. If you would like to use mathtime but you are seeing error
% messages about missing fonts (mtex.pfb, mtsy.pfb, or rmtmi.pfb) then please see
% the technical support document at http://www.aeaweb.org/templates/technical_support.pdf
% for instructions on fixing this problem.

% Note: you may use either harvard or natbib (but not both) to provide a wider
% variety of citation commands than latex supports natively. See below.

% Uncomment the next line to use the natbib package with bibtex
\usepackage{natbib}

% Uncomment the next line to use the harvard package with bibtex
%\usepackage[abbr]{harvard}

% This command determines the leading (vertical space between lines) in draft mode
% with 1.5 corresponding to "double" spacing.
\draftSpacing{1.5}

\usepackage{hyperref}

\begin{document}

\title{Agricultural Land Rents Reflect Climate Change Impacts}
\shortTitle{Climage Change Impacts on Agriculture}
% \author{Author1 and Author2\thanks{Surname1: affiliation1, address1, email1.
% Surname2: affiliation2, address2, email2. Acknowledgements}}


\author{
  Nicholas A. Potter\thanks{
  Potter: Washington State University, \href{mailto:nicholas.a.potter@wsu.edu}{nicholas.a.potter@wsu.edu}.
}
}

\date{\today}
\pubMonth{7}
\pubYear{2017}
\pubVolume{1}
\pubIssue{1}
\JEL{Q10, Q54}
\Keywords{Climate, Agriculture}

\begin{abstract}
``Climate change affects the value of agricultural output through
different mechanisms. Increases in temperature reduce yields on the
right side of the temperature-yield curve. Warmer and longer growing
seasons increase yields. Using a Ricardian framework in which farmers
adapt to changing conditions to maximize profits, we match projected
climate with the closest historical climate analogue for a
latitude-longitude grid of the United States to estimate the effect of
climate change on agriculture. The difference in land rents between each
match is an estimate of change due to climate. Using this method,
climate in 2040 will increase/reduce agricultural land rent values by
XXX across the U.S., with substantial variation by county. These results
support the idea that XXX and are at odds with research suggesting that
XXX.''
\end{abstract}


\maketitle

How will changes in climate affect agricultural land rents?

\section{Effects on Yields}\label{effects-on-yields}

Lobell, Schlenker, and Costa-Roberts
(\protect\hyperlink{ref-LobellSchlenkerCosta-Roberts2011}{2011})

Gelman and Price (\protect\hyperlink{ref-GelmanPrice1999}{1999})

Attempts to estimate the economic impacts of changing climate fall into
two camps. One examines agricultural yields based on expected
temperatures, noting that yields increase with temperature up to a
point, then decrease rapidly. If farmers are unable to switch crops or
make other adjustments for a changing climate, they will be left with
less productive land. Schlenker and Roberts
(\protect\hyperlink{ref-SchlenkerRoberts2009}{2009}) estimate that
yields will decrease between 30-46\% under the slowest warming scenario
and by 63-82\% under the fastest warming scenario. To some extent this
loss in yields may be partially offset by increased CO2 levels (Long et
al. \protect\hyperlink{ref-LongETAL2006}{2006}).

Others such as Mendelsohn, Nordhaus, and Shaw
(\protect\hyperlink{ref-MendelsohnNordhausShaw1994}{1994}) allow farmers
to adjust to climate change as they best see fit, instead using a
`Ricardian approach' that assumes well-functioning markets in which the
value of yields is captured in the value of land. To use the example
from MNS, as temperatures rise wheat farmers may switch to corn, and the
value of their land will reflect that new use.

Mendelsohn and Nordhaus
(\protect\hyperlink{ref-MendelsohnNordhaus1999}{1999})

Darwin (\protect\hyperlink{ref-Darwin1999}{1999})

Schlenker, Hanemann, and Fisher
(\protect\hyperlink{ref-SchlenkerHanemannFisher2006}{2006})

Mendelsohn et al. (\protect\hyperlink{ref-MendelsohnETAL2000}{2000})

Mendelsohn, Schlesinger, and Williams
(\protect\hyperlink{ref-MendelsohnSchlesingerWilliams2000}{2000})

Hope (\protect\hyperlink{ref-Hope2006}{2006})

Tol
(\protect\hyperlink{ref-Tol2002a}{2002}\protect\hyperlink{ref-Tol2002a}{b})
Tol
(\protect\hyperlink{ref-Tol2002b}{2002}\protect\hyperlink{ref-Tol2002b}{a})

Potter and Adams (\protect\hyperlink{ref-rnassqs}{2017})

Soil Survey Staff (\protect\hyperlink{ref-SSURGO}{2017})

Recht (\protect\hyperlink{ref-censusapi}{2017})

The Ricardian approach is appealing since it allows farmers to adapt to
changing conditions, but it has its limitations. Farmers may be limited
in their ability to adapt either by a lack of capital or a dearth of
solutions.

At issue is not just warming temperatures and changing levels of
precipitation. Climate is also expected to become more variable.
Increased fluctuation in temperatures can significantly impact yields,
even if average temperatures increase a few degrees. Fluctuation in
precipitation may mean periods of drought followed by heavy rains, which
will also affect yields.

In addition to this, expectations of increased climate variability will
likely negatively impact yields of a wide variety of crops.

Sekhon (\protect\hyperlink{ref-Sekhon2011}{2011})

Yang and Shumway (\protect\hyperlink{ref-YangShumway2015}{2015})

\section{Ricardian Approach}\label{ricardian-approach}

\section{Land Rents}\label{land-rents}

\section{Farmers Account for Climate Change in Current Land
Values}\label{farmers-account-for-climate-change-in-current-land-values}

\section{American Economic Review
Pointers}\label{american-economic-review-pointers}

\begin{itemize}
\item Do not use an "Introduction" heading. Begin your introductory material
before the first section heading.

\item Avoid style markup (except sparingly for emphasis).

\item Avoid using explicit vertical or horizontal space.

\item Captions are short and go below figures but above tables.

\item The tablenotes or figurenotes environments may be used below tables
or figures, respectively, as demonstrated below.

\item If you have difficulties with the mathtime package, adjust the package
options appropriately for your platform. If you can't get it to work, just
remove the package or see our technical support document online (please
refer to the author instructions).

\item If you are using an appendix, it goes last, after the bibliography.
Use regular section headings to make the appendix headings.

\item If you are not using an appendix, you may delete the appendix command
and sample appendix section heading.

\item Either the natbib package or the harvard package may be used with bibtex.
To include one of these packages, uncomment the appropriate usepackage command
above. Note: you can't use both packages at once or compile-time errors will result.

\end{itemize}

Sample figure:

\begin{figure}
Figure here.

\caption{Caption for figure below.}
\begin{figurenotes}
Figure notes without optional leadin.
\end{figurenotes}
\begin{figurenotes}[Source]
Figure notes with optional leadin (Source, in this case).
\end{figurenotes}
\end{figure}

Sample table:

\begin{table}
\caption{Caption for table above.}

\begin{tabular}{lll}
& Heading 1 & Heading 2 \\
Row 1 & 1 & 2 \\
Row 2 & 3 & 4%
\end{tabular}
\begin{tablenotes}
Table notes environment without optional leadin.
\end{tablenotes}
\begin{tablenotes}[Source]
Table notes environment with optional leadin (Source, in this case).
\end{tablenotes}
\end{table}

\section*{References}\label{references}
\addcontentsline{toc}{section}{References}

\hypertarget{refs}{}
\hypertarget{ref-Darwin1999}{}
Darwin, Roy. 1999. ``The Impact of Global Warming on Agriculture: A
Ricardian Analysis: Comment.'' \emph{The American Economic Review} 89
(4). JSTOR: 1049--52.

\hypertarget{ref-GelmanPrice1999}{}
Gelman, Andrew, and Phillip N Price. 1999. ``All Maps of Parameter
Estimates Are Misleading.'' \emph{Statistics in Medicine} 18 (23):
3221--34.

\hypertarget{ref-Hope2006}{}
Hope, Chris W. 2006. ``The Marginal Impacts of Co2, Ch4 and Sf6
Emissions.'' \emph{Climate Policy} 6 (5). Taylor \& Francis: 537--44.

\hypertarget{ref-LobellSchlenkerCosta-Roberts2011}{}
Lobell, David B, Wolfram Schlenker, and Justin Costa-Roberts. 2011.
``Climate Trends and Global Crop Production Since 1980.'' \emph{Science}
333 (6042). American Association for the Advancement of Science:
616--20.

\hypertarget{ref-LongETAL2006}{}
Long, Stephen P, Elizabeth A Ainsworth, Andrew DB Leakey, Josef
Nösberger, and Donald R Ort. 2006. ``Food for Thought:
Lower-Than-Expected Crop Yield Stimulation with Rising Co2
Concentrations.'' \emph{Science} 312 (5782). American Association for
the Advancement of Science: 1918--21.

\hypertarget{ref-MendelsohnNordhaus1999}{}
Mendelsohn, Robert, and William D Nordhaus. 1999. ``The Impact of Global
Warming on Agriculture: A Ricardian Analysis: Reply.'' \emph{The
American Economic Review} 89 (4). JSTOR: 1046--8.

\hypertarget{ref-MendelsohnETAL2000}{}
Mendelsohn, Robert, Wendy Morrison, Michael E Schlesinger, and Natalia G
Andronova. 2000. ``Country-Specific Market Impacts of Climate Change.''
\emph{Climatic Change} 45 (3). Springer: 553--69.

\hypertarget{ref-MendelsohnNordhausShaw1994}{}
Mendelsohn, Robert, William D Nordhaus, and Daigee Shaw. 1994. ``The
Impact of Global Warming on Agriculture: A Ricardian Analysis.''
\emph{The American Economic Review} 84 (4). JSTOR: 753--71.

\hypertarget{ref-MendelsohnSchlesingerWilliams2000}{}
Mendelsohn, Robert, Michael Schlesinger, and Larry Williams. 2000.
``Comparing Impacts Across Climate Models.'' \emph{Integrated
Assessment} 1 (1). Springer: 37--48.

\hypertarget{ref-rnassqs}{}
Potter, Nicholas A, and Jonathan Adams. 2017. ``Rnassqs: An R Library to
Access the Usda Nass Quickstats Api.''
doi:\href{https://doi.org/10.5281/zenodo.825265}{10.5281/zenodo.825265}.

\hypertarget{ref-censusapi}{}
Recht, Hannah. 2017. ``Censusapi: R Package to Retrieve Census Data and
Metadata via Api.'' \url{https://github.com/hrecht/censusapi}.

\hypertarget{ref-SchlenkerRoberts2009}{}
Schlenker, W., and M. J. Roberts. 2009. ``Nonlinear Temperature Effects
Indicate Severe Damages to U.S. Crop Yields Under Climate Change.''
\emph{Proceedings of the National Academy of Sciences} 106 (37).
Proceedings of the National Academy of Sciences: 15594--8.
doi:\href{https://doi.org/10.1073/pnas.0906865106}{10.1073/pnas.0906865106}.

\hypertarget{ref-SchlenkerHanemannFisher2006}{}
Schlenker, Wolfram, W Michael Hanemann, and Anthony C Fisher. 2006.
``The Impact of Global Warming on Us Agriculture: An Econometric
Analysis of Optimal Growing Conditions.'' \emph{The Review of Economics
and Statistics} 88 (1). MIT Press: 113--25.

\hypertarget{ref-Sekhon2011}{}
Sekhon, Jasjeet S. 2011. ``Multivariate and Propensity Score Matching
Software with Automated Balance Optimization: The Matching Package for
R.'' \emph{Journal of Statistical Software} 42 (7): 1--52.

\hypertarget{ref-SSURGO}{}
Soil Survey Staff, United States Department of Agriculture, Natural
Resources Conservation Service. 2017. ``Web Soil Survey.''
\url{https://websoilsurvey.nrcs.usda.gov/}.

\hypertarget{ref-Tol2002b}{}
Tol, Richard SJ. 2002a. ``Estimates of the Damage Costs of Climate
Change, Part Ii. Dynamic Estimates.'' \emph{Environmental and Resource
Economics} 21 (2). Springer: 135--60.

\hypertarget{ref-Tol2002a}{}
---------. 2002b. ``Estimates of the Damage Costs of Climate Change.
Part 1: Benchmark Estimates.'' \emph{Environmental and Resource
Economics} 21 (1). Springer: 47--73.

\hypertarget{ref-YangShumway2015}{}
Yang, Sansi, and C. Richard Shumway. 2015. ``Dynamic Adjustment in Us
Agriculture Under Climate Change.'' \emph{American Journal of
Agricultural Economics} 98 (3): 910--24.
doi:\href{https://doi.org/10.1093/ajae/aav042}{10.1093/ajae/aav042}.

\end{document}

